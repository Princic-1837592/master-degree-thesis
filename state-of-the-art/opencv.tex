\section{OpenCV}\label{sec:opencv}
OpenCV, which stands for Open Source Computer Vision Library,
is an open source library dedicated to image processing and computer vision.
Originally developed by Intel in 1999~\cite{learning-opencv}, OpenCV has grown considerably over the years, becoming an
essential tool in various fields, from robotics to augmented reality, from surveillance to aerial imaging~\cite{learning-opencv}.

\subsection{Origin and history}\label{subsec:origin-and-history}
The story of OpenCV began in the late 1990s, when Intel started the project as part of an initiative
to promote the use of CPUs in artificial vision applications.
In 2000, OpenCV was released to the public in its alpha version.
It then evolved through 5 beta versions between 2000 and 2005, finally reaching its first official release
version 1.0 in 2006~\cite{learning-opencv}.
The library is now at version 4.10.0, and on GitHub it has 63 releases and almost 77000 stars.

\subsection{Library overview}\label{subsec:library-overview}
OpenCV is written in \texttt{C} and \texttt{C++}~\cite{learning-opencv}, but offers bindings for other programming
languages such as \texttt{Python}, \texttt{Java}, \texttt{C\#}
and \texttt{MATLAB}, making it accessible to a wide range of developers.
The library contains over 2500 optimised algorithms covering different areas of image processing and computer vision, including:
\begin{itemize}
	\item Object detection
	\item Facial recognition
	\item Gesture recognition
	\item Image segmentation
	\item Motion tracking
	\item 3D reconstruction
	\item Video processing
\end{itemize}

\subsection{Main applications}\label{subsec:main-applications}

\paragraph{Image and Video Processing}
One of the most common uses of OpenCV is image and video processing.
The library provides tools for manipulating, transforming and analysing images in various formats.
These tools include noise reduction filters, geometric transformations, edge detection, and many other image enhancement techniques.

\paragraph{Object and Face Recognition}
OpenCV is widely used for object and face recognition~\cite{opencv-face-recognition-1, opencv-face-recognition-2, opencv-face-recognition-3}.
Algorithms such as Haar Cascade, LBP (Local Binary Patterns) and OpenCV's DNN (Deep Neural Network) technology make it
possible to detect and recognise faces and other objects in images with high accuracy.
These algorithms can be used in security applications, such as surveillance systems, and in social media apps for adding face filters.

\paragraph{Augmented Reality}
In the field of augmented reality, OpenCV can be used to render digital information on top of the real world.
OpenCV's tracking and sensing algorithms enable the detection and tracking of physical objects, such as AR
markers~\cite{opencv-marker-1, opencv-marker-2, opencv-marker-3}, which serve as reference points for overlaying virtual content.
This capability is crucial for creating immersive and interactive experiences.

\paragraph{Robotics}
In robotics, OpenCV is used to equip robots with ``vision''~\cite{opencv-robotics-1, opencv-robotics-2, opencv-robotics-3}.
Through the library and its functions listed above, robots can perceive and understand their surroundings,
identify obstacles, detect visual signals and follow predefined paths.
These capabilities are essential for the development of autonomous robots and drones.

\paragraph{Medical Applications}
OpenCV also has applications in the medical sector, where it is involved in the analysis of medical images such as
X-rays~\cite{opencv-x-ray}, MRI~\cite{opencv-mri-1, opencv-mri-2, opencv-mri-3}
and ultrasound images~\cite{opencv-ultrasound-1, opencv-ultrasound-2, opencv-ultrasound-3, opencv-ultrasound-4}.
The automatic analysis of medical images helps doctors diagnose diseases more accurately and quickly.
