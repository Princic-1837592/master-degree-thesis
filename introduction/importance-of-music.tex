\section{Importance of music and its education}\label{sec:importance-of-music-and-its-education}
Music has always been an important component in the culture of human civilizations.
It is one of the oldest and most widespread art forms,
traces of which can be found as far back as the Paleolithic~\cite{origin-of-music},
and has enabled entire peoples to preserve and pass on traditions, share ideas and unite people.

Music education has a rich history that can be traced back to ancient Greece,
where melody was combined with poetry and dance, creating an almost inseparable whole.
Here, music was considered a gift from the muses, gods of the arts, and was thought to have the power
to influence people's thoughts and actions in various ways, and different instruments were associated with different
effects on the emotions and character of individuals~\cite{music-education-in-ancient-greece}.

According to Plato, musical education starts from early childhood,
when mothers should sing to and dance with their children.
They, in fact, unable to speak and understand complex thoughts, can nevertheless appreciate melody and singing,
through which they can be educated and instructed.
The period of formal education, again according to Plato, should start at the age of 6 and last until 20,
focusing on dance, music and gymnastics~\cite{music-education-in-ancient-greece}.

In Europe, music is part of the school curriculum in almost all countries.
According to a research carried out by \textit{meNet} from 2006 to 2009, it emerged that in most European countries,
music is treated similarly in similar age groups: first, it is offered as an optional subject in pre-schools,
and becomes a compulsory subject during primary school in all countries.
It remains compulsory in almost all countries at least up to the age of 14, which corresponds more or less to middle school.

The Czech Republic, Hungary, the Netherlands and Slovakia have the longest compulsory period,
which goes from the age of 5 to the age of 18.

In Italy, music education in schools has a history of more than a hundred years~\cite{storia-educazione-musica}.
Starting out as an optional subject called ``Canto corale'' (``Choral Singing'') in 1888,
music was regarded as a subject of low importance, more useful for recreation than education, for 35 years.

It was in 1923 that the subject ``Canto'' (``Singing'') was elevated to a compulsory subject for elementary schools,
in the group of ``Insegnamenti artistici'' (``Art Teaching'') along with ``Disegno spontaneo'' (``Spontaneous Drawing''),
located first in the order of subjects.

In the 1959--60 school year, the experiment to make the teaching of music compulsory in all three years of middle school was launched.
The experiment failed and we will have to wait until 1977 to see the subject ``Educazione musicale'' (``Music Education'')
become compulsory in all three years of middle school.

Finally, it is with a reform in 2010 that a high school curriculum focusing on the teaching and practice of music,
called ``Liceo musicale'' (``Music High School''), is established.

%spostare sotto
%and our goal with this project is to simplify and make more accessible, specifically,
%the teaching of the piano through an augmented reality app, the purpose of which is to cut costs related to
%the purchase of instruments and the burden of transport (for the students), or storage (for the school).
