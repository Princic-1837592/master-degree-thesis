\section{VR and AR for music education}\label{sec:vr-in-education}
In recent years, especially during and after the COVID--19 pandemic,
there has been an increase in the use of technology in schools for management and educational purposes.
Among other technologies that have entered the world of education, virtual reality and augmented reality represent
two innovative technological tools that can transform the way teaching and learning are conducted in schools.
Both technologies offer new opportunities to enrich the educational experience,
making learning more interactive, immersive and engaging,
and giving students the opportunity to experience authentic situations and apply their knowledge in practice.

Music education and learning an instrument are challenges for both the student and the teacher,
also due to budget cuts and little emphasis on arts subjects in schools.
Using VR or AR in music education can be an alternative approach to improve both the learning and teaching experience,
and to bring attention back to this too often undervalued subject.

Augmented reality and mixed reality, in particular, have the added advantage over virtual reality of bringing
the physical and digital worlds into direct contact through objects that exist
in both forms and enable more realistic experiences with more immediate feedback.

The potential advantages of augmented reality teaching methods over traditional methods include:
learning and practicing rhythmic skills, being able to play together while being apart, overcoming stage fear, and 
training STEAM (Science, Technology, Engineering, Art and Math) and acoustic skills~\cite{vr-and-ar-in-music-education}
