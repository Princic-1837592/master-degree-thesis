\section{Unity}\label{sec:unity}
Unity is a professional, robust and versatile game development platform
that has gained widespread popularity~\cite{unity-technical-survey} for its ability to create both 2D and 3D applications.
The Unity editor is cross-platform, being supported on Windows, macOS and Linux, and its engine supports builds
for more than 19 different platforms, including mobile and desktop devices, consoles and VR devices.

Developed by Unity Technologies, it has evolved into a powerful tool for implementing
high performance applications~\cite{unity-interview-high-speed}, used by developers worldwide for a variety of projects,
ranging from games to simulations, augmented reality~\cite{marco-ar, marco-ar-mano},
and virtual reality~\cite{marco-vr} applications.

Augmented reality applications made with Unity can easily be used in a
classroom~\cite{ar-application-in-classroom-taxonomy, ar-for-preschool-children, vr-and-ar-for-learning, educational-ar}
as an innovative and interactive teaching method for children, with examples of its usage in the teaching of subjects such as
Astronomy~\cite{ar-astronomy}, Volcanology~\cite{ar-volcanology}, Engineering~\cite{ar-engineering},
Medicine~\cite{ar-medicine} and more~\cite{ar-microeconomics, ar-animals}.
This makes Unity the perfect tool for our project.

Our application requires the use of the OpenCV and MediaPipe libraries, and is written entirely in \texttt{C\#} within the Unity Engine.
This led us to the adoption of the corresponding plugins within the game engine.
The exact versions of Unity and plugins used within the project are:
\begin{itemize}
	\item Unity Editor: 2022.3.18f1
	\item OpenCV for Unity: 2.5.8
	\item MediaPipe for Unity: 0.14.1
\end{itemize}

\subsection{OpenCV}\label{subsec:opencv-for-unity}
OpenCV for Unity is a very popular plugin for using OpenCV within the Unity development environment.
It can be purchased from the Unity Asset Store and easily imported into a project.
From there, its use differs little from what it would be in \texttt{C++} or \texttt{Python}:
it supports practically all the functions and constants present in the original module,
while maintaining an API identical to its \texttt{C++} counterpart.

The support for WebCamTexture makes OpenCV for Unity easy to use in applications that need to perform real-time image processing,
which makes it an excellent choice for our project.

\subsection{MediaPipe}\label{subsec:mediapipe-for-unity}
MediaPipe for Unity is a native plugin which ports the \texttt{C++} MediaPipe API to \texttt{C\#}, one by one,
so that it can easily be used within the Unity development environment.
It can be installed from the documentation page and be directly imported into the project.
Thanks to the precise documentation, the various example scenes and the clear and comprehensive tutorial,
using MediaPipe in Unity is quick and easy.

This library supports all major desktop operating systems (Linux, macOS and Windows) as well as the mobile systems
Android and iOS, for which GPU mode can also be enabled, which is why we chose to use this library in our project.
