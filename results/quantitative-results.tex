\section{Quantitative results}\label{sec:quantitative-results}
The experiments carried out on the application concern the two fundamental
phases of the pipeline: the keyboard detection phase and the real-time phase.

\subsection{Keyboard detection}\label{subsec:keyboard-detection}
Experiments to evaluate the keyboard detection algorithm were carried out on 3 different types of keyboard:
a hand-drawn keyboard, a keyboard drawn with a ruler as the one shown in~\autoref{fig:screenshot-detection-phase},
and a keyboard printed using the preset provided by the application shown in~\autoref{fig:preset-1}.

The tests were carried out in a room with artificial light coming from the ceiling
and natural light coming from an open window.
For each of these keyboards, the test was carried out by changing the rotation and position of
the keyboard with respect to the natural light 4 times, performing the keyboard detection 3 times for each position,
for a total of 12 detections per keyboard type and 36 detections in total.

The evaluation of the experiments consists of comparing the result obtained from the keyboard detection, such as the
one shown in~\autoref{fig:tiles-overlay}, with the optimal result we would expect to obtain from a perfect detector.
Of course, we are not able to obtain the optimal result with an algorithm,
otherwise we would use that same algorithm to do the detection.
So to create the ground truth we used a photo editing software to manually colour
in black and white exactly the spaces occupied by the keys.

Once we had created the ground truth for all 36 photos,
we applied the keyboard detection algorithm to each of them and compared it to the ground truth
by doing a simple subtraction of colours to highlight individual differing pixels.
To calculate the pixel accuracy of our detection algorithm we used the following formula
\[
	Accuracy = \frac{TP}{n}
\]
where $TP$ (true positives) is the number of pixels correctly classified by the algorithm and $n$ is the total number
of pixels within the area bounded by the perimeter, as can be seen in~\autoref{fig:screenshot-detection-phase}.
We used the number of pixels inside the perimeter and not the total number of pixels in the image because doing so
would have produced higher but untrue percentages, as the algorithm only searches for the keyboard in
the area of the image bounded by the perimeter and not in the entire image.

As can be seen in~\autoref{tab:keyboard-detection-accuracy},
the results for the detection algorithm are excellent for each type of keyboard.

\begin{table}[ht]
	\centering
	\begin{tabular}{l|l|}
		\cline{2-2}
		& Pixel accuracy               \\ \hline
		\multicolumn{1}{|l|}{Hand-drawn}       & \multicolumn{1}{c|}{95.00\%} \\ \hline
		\multicolumn{1}{|l|}{Drawn with ruler} & \multicolumn{1}{c|}{96.74\%} \\ \hline
		\multicolumn{1}{|l|}{Printed}          & \multicolumn{1}{c|}{97.69\%} \\ \hline
	\end{tabular}
	\caption{Pixel accuracy of keyboard detection algorithm}
	\label{tab:keyboard-detection-accuracy}
\end{table}

Unfortunately, we did not conduct the same experiment for the keyboard detection algorithm in the old prototype,
so we cannot compare the results.
However, based on our experience with both algorithms,
we can state with absolute certainty that the new algorithm is much better performing than the old one,
which was only able to detect straight lines, needed the additional step of background subtraction,
and didn't even produce a visible result like the new one does.

\subsection{Real-time phase}\label{subsec:real-time-phase-results}
To evaluate the real-time phase we used a method similar to the one used in the old prototype:
we played a series of notes on the keyboard and counted the times in which the application reacted correctly or not.

In order to evaluate the performance of the application's real-time phase according to its accuracy,
precision, and recall, we needed to calculate the amount of true positives ($TP$), true negatives ($TN$),
false positives ($FP$) and false negatives ($FN$).
The meaning of these terms in our application is schematized in~\autoref{tab:tp-tn-fp-fn}.

\begin{table}[ht]
	\centering
	\begin{tabular}{l|l|l|}
		\cline{2-3}
		& User plays                & User does not play        \\ \hline
		\multicolumn{1}{|l|}{Application plays}         & \multicolumn{1}{c|}{$TP$} & \multicolumn{1}{c|}{$FP$} \\ \hline
		\multicolumn{1}{|l|}{Application does not play} & \multicolumn{1}{c|}{$FN$} & \multicolumn{1}{c|}{$TN$} \\ \hline
	\end{tabular}
	\caption{Meaning of $TP$, $TN$, $FP$ and $FN$ in our application}
	\label{tab:tp-tn-fp-fn}
\end{table}

To achieve the maximum precision in calculating these values, we modified the real-time algorithm to be able to save
each frame associated with a label indicating whether in that frame the algorithm detected a touch or not.
After the recording, we manually checked each frame with the corresponding label and counted if it was a $TP$, $TN$, $FP$ or $FN$.

In this way we carried out 3 experiments.
In each experiment the user had to press a finger on all the keys of the keyboard in order from left to right.
The results of the experiments are shown in~\autoref{tab:real-time-results},
together with the results from the old prototype.

\begin{table}[ht]
	\centering
	\begin{tabular}{l|l|l|l|}
		\cline{2-4}
		& Accuracy                      & Precision                     & Recall                        \\ \hline
		\multicolumn{1}{|l|}{Old prototype} & \multicolumn{1}{|c|}{90.27\%} & \multicolumn{1}{|c|}{56.87\%}   & \multicolumn{1}{|c|}{99.40\%} \\ \hline
		\multicolumn{1}{|l|}{New version}   & \multicolumn{1}{|c|}{87.40\%} & \multicolumn{1}{|c|}{80.23\%}   & \multicolumn{1}{|c|}{81.18\%} \\ \hline
	\end{tabular}
	\caption{Comparison of real-time phase between old prototype and current version}
	\label{tab:real-time-results}
\end{table}

The comparison between the old and the new version highlights how the real-time phase of the application is now much
more precise than before, and on average more stable and reliable.
The recall of the new version is much lower than before, but this is because in the old version there were very few
false negatives thanks to the presence of the Leap Motion Controller.
