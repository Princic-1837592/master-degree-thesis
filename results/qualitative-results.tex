\section{Qualitative results}\label{sec:qualitative-results}
We also retrieved results about the perceived usability of our application with the same method used for the old prototype.
We asked 15 participants to play with the application for 5 minutes,
from the keyboard detection phase to the playing phase, and to fill out the UMUX questionnaire
(Usability Metric for User Experience~\cite{umux}) at the end.
Among the participants involved in the experiment, there were people with different musical background knowledge:
6 had no musical knowledge at all, 6 had little musical knowledge mainly from middle school and high school,
and 3 were professional piano players.

The results of the questionnaire are shown in~\autoref{tab:usability-results},
together with the results from the old prototype.

\begin{table}[ht]
	\centering
	\begin{tabular}{l|l|l|l|}
		\cline{2-4}
		& Minimum                       & Average                       & Maximum                       \\ \hline
		\multicolumn{1}{|l|}{Old prototype} & \multicolumn{1}{|c|}{70.83\%} & \multicolumn{1}{|c|}{79.86\%} & \multicolumn{1}{|c|}{87.50\%} \\ \hline
		\multicolumn{1}{|l|}{New version}   & \multicolumn{1}{|c|}{84.11\%} & \multicolumn{1}{|c|}{92.28\%} & \multicolumn{1}{|c|}{97.91\%} \\ \hline
	\end{tabular}
	\caption{Comparison of usability between old prototype and current version}
	\label{tab:usability-results}
\end{table}

These results highlight how the porting from computer to smartphone,
the elimination of all additional hardware, and the simplification of the keyboard detection procedure
have greatly improved the usability and portability of the entire system.
